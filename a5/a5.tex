\documentclass[]{article}

\usepackage[]{geometry}
\geometry{
  top=1in,            % <-- you want to adjust this
  inner=1in,
  outer=1in,
  bottom=1in,
  headheight=3ex,       % <-- and this
  headsep=2ex,          % <-- and this
}
\usepackage[T1]{fontenc}
\usepackage{cmbright}
\usepackage{mathtools}
\usepackage{algorithmic}
\usepackage{fancyhdr}
\usepackage{amssymb}
\usepackage{multicol}
\usepackage{parskip}
\usepackage{titling}
\pretitle{\begin{flushleft}\LARGE\sffamily}
\posttitle{\par\end{flushleft}\vskip 0.5em}
\preauthor{\begin{flushleft}}
\postauthor{\par\end{flushleft}}
\predate{\begin{flushleft}\scshape}
\postdate{\par\end{flushleft}}
\setlength{\droptitle}{-20pt}
\setlength{\headheight}{15.2pt}
\pagestyle{fancy}

\fancyhf{}
\renewcommand{\headrulewidth}{0pt} % remove line at top
\lhead{\fancyplain{}{CS 4780 A5}}
\rhead{\fancyplain{}{Justin Cheng \emph{jc882} and Sunling Selena Yang (sy???)}}
\rfoot{\fancyplain{}{\thepage}}

\begin{document}

\title{CS 4780 Assignment 5}
\author{Justin Cheng and Sunling Selena Yang}
\date{\today}
\maketitle

\hrule
\vskip 1em

\section{Viterbi Algorithm}

\subsection{Most Likely Translations}

What follows in each section is the table of probabilities for partial paths, then the back-pointer tables.

\subsubsection{$\alpha \eta$}

\begin{tabular}{ |c|c|c| }
\hline
  & $\alpha$ & $\eta$ \\
\hline
a & 0.04 & 0.0042 \\
t & 0.03 & 0.0024 \\
o & 0.02 & 0.006 \\
n & 0.12 & 0.004 \\
\hline
\end{tabular}

\begin{tabular}{ |c|c|c| }
\hline
  & $\alpha$ & $\eta$ \\
\hline
a & None & n \\
t & None & a \\
o & None & n \\
n & None & o \\
\hline
\end{tabular}

Predicting "no".

\subsubsection{$\tau \omega \gamma$}

\begin{tabular}{ |c|c|c|c| }
\hline
  & $\tau$ & $\omega$ & $\gamma$ \\
\hline
a & 0.02 & 0.0048 & 0.00048 \\
t & 0.12 & 0.0012 & 0.00216 \\
o & 0.02 & 0.024 & 0.00048 \\
n & 0.04 & 0.0012 & 0.0012 \\
\hline
\end{tabular}

\begin{tabular}{ |c|c|c|c| }
\hline
  & $\tau$ & $\omega$ & $\gamma$ \\
\hline
a & None & t & o \\
t & None & a & o \\
o & None & t & o \\
n & None & t & o \\
\hline
\end{tabular}

Predicting "tot".

\subsubsection{$\gamma \alpha \omega \eta$}

\begin{tabular}{ |c|c|c|c|c| }
\hline
  & $\gamma$ & $\alpha$ & $\omega$ & $\eta$\\
\hline
a & 0.02 & 0.0144 & 0.00021 & 3.456e-05 \\
t & 0.09 & 0.0012 & 0.000864 & 5.4e-05 \\
o & 0.04 & 0.0036 & 0.0018 & 3.456e-05 \\
n & 0.04 & 0.006 & 0.00018 & 0.00036 \\
\hline
\end{tabular}

\begin{tabular}{ |c|c|c|c|c| }
\hline
  & $\gamma$ & $\alpha$ & $\omega$ & $\eta$\\
\hline
a & None & t & n & t \\
t & None & a & a & o \\
o & None & t & a & t \\
n & None & o & a & o \\
\hline
\end{tabular}

Predicting "taon".

\subsection{Complexity}
In each "step" of the dynamic program, perform $m$ lookups for each of the English letters. For each English letter, compare against previous $m$ English letters, for a total cost of $O(m^2)$.

There are $k$ steps. So the total cost is $O(km^2)$. Notice that this is independent of the size of the Emelic vocabulary.

In a brute-force algorithm, try all possibilities. So for each Emelic letter, try $m$ English letters. Since there are $k$ letters, this is $O(m^k)$.

\subsection{Probability of an observation}

First we calculate 

$P(x_i) = \sum_{y_i} P(x_i \cap y_i) = \sum_{y_i} P(x_i|y_i)P(y_i)$, and we know the values of both $P(x_i|y_i)$ from table 2 and $P(y_i)$ from START in table 1.

Also, $P(y_i|x_i) = P(x_i|y_i)P(y_i)/P(x_i)$.

Then, $P(x_i|x_{i-1}) = P(x_i \cap x_{i-1}) / P(x_{i-1}) = \frac{\sum_{y_i} P(x_i|y_i)P(y_i|y_{i-1})P(x_{i-1}|y_{i-1})}{P(x_{i-1})}$. It is easier to visualize this specific probability by thinking about the state diagram with nodes corresponding to $y_i$s, arrows corresponding to $P(y_i|y_{i-1})$ and "pies" at each node corresponding to $P(x_i|y_i)$.

Now, $P(x_1,...,x_k) = \sum_{all~ys} P(x_1,...,x_k \cap y_1,...,y_k) = \sum_{all~ys}P(x_1,...,x_k|y_1,...,y_k)P(y_1,...,y_k)$

And $P(x_1,...,x_k|y_1,...,y_k) = P(x_1)P(y_1|x_1)\prod_{i=2}^{i=k} P(y_i|x_i)P(x_i|x_{i-1})$

But we know $P(x_i)$, $P(y_i|x_i)$, and $P(x_i|x_{i-1})$ and $P(y_1,...,y_k) = P(y_1)P(y_2|y_1)...$

Thus we can then calculate $P(x_1,...,x_k)$ from the above equation.

\subsection{Better translation model}

It would do badly vs the state-of-the-art, because of the different grammars of the two languages. For example, if in one language the sentence structure is Subject-Verb-Object and in the other it is Subject-Object-Verb, it would be difficult to use HMM to derive the right transitions because of the altered ordering.

\section{Statistical Learning Theory}

\subsection{Restricted Linear Classifier}

With probability $(1-\delta)$, $Err_p(h_{L(S)}) \le Err_s(h_{L(S)}) + \sqrt{\frac{1}{2n}(\log(2|H|) - \log(\delta))}$

Now we know that $Err_s(h_{L(S)}) = 0$, all that remains is finding $|H|$.

Because each of the 100 weight vectors are binary, and the bias can take 21 values, $|H| = 2^{100} \cdot 21$.

Then $Err_p(h_{L(S)}) \le \sqrt{\frac{1}{2n} \log(2^{101}\cdot 21 - \log \delta)} = \sqrt{\frac{1}{2n}(101 \log 2 + \log 21-\log \delta)}$.

\subsection{Unrestricted Linear Classifier}
Because there is now no restriction on $w_i$ and $b$, this is a linear classifier in $\mathbb{R}^{100}$, so VCDim = 101.

By Sauer's Lemma, $|\pi_H(S)| \le (\frac{e\cdot |S|}{VCDim(H)})^{VCDim(H)} = (\frac{en}{101})^{101}$.

Then $Err_p(h_{L(S)}) \le Err_s(h_{L(S)}) + \sqrt{\frac{1}{2n}(\log(2|H|) - \log(\delta))} = \sqrt{\frac{1}{2n}(101\log(\frac{2en}{101}) - \log(\delta))}$.

\subsection{Training Set Size}

By the Union and Chernoff Bounds, we need to find out $n=|S|$ such that $|H|2e^{-2n\epsilon^2} \le \delta$

or that $(2^{100} \cdot 21)\cdot 2 e^{-2n(0.1)^2} \le 0.1$

Solving for $n$ gives us $n \ge 3767.74...$ so $n$ should be 3768.

\subsection{Spherical Classifiers}

\subsubsection{At least $d$}
Consider the example set of $d$ elements in $d$ dimensions, where the coordinate of each element is $1$ for one unique dimension and $0$ everywhere else. For example, the first element has coordinates $(1,0,0,...)$ and the second $(0,1,0,0,0...)$

Any classification of these $d$ elements is linearly separable, and we can simply draw a hyperplane to separate these points. We can now construct an arbitrarily large n-dimensional sphere tangent to this hyperplane on whichever side is required to be positive. If the sphere is large enough, we will cover all points on one side of the hyperplane.

\subsubsection{No more than $2d+1$}
Note that $R - ||x-\vec{c}|| = R - \sqrt{x_1^2 + ... + x_d^2 + \vec w^T x + b}$ where $\vec{w} = (-2c_1,...,-2c_d)$ and $b = c_1^2 + ... + c_d^2$.

Ignoring constants, we now have the equation for a linear classifier and $d$ other terms. A linear classifier in $d$ dimensions can classify up to $d+1$ points. These $d$ other terms can each only classify up to $1$ additional term.

Thus the VCdim is $d+1 + d = 2d+1$.

\end{document}